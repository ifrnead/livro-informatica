\chapter{Introdução}
\label{Cap:introducao}

A riqueza de problemas complexos de otimização encontrados no mundo real tais como telecomunicações, logística, transporte e planejamento financeiro, tem demonstrado que sua solução não é uma tarefa fácil, pois existem inúmeras situações em que é impossível se construir um modelo detalhado para o problema, dada sua elevada complexidade. Por outro lado, um processo de simplificação de tal modelo pode causar perdas de informações relevantes que comprometem  a qualidade dessa solução. Além da dificuldade inerente à construção de modelos para tais problemas, uma  característica que os acompanha durante a fase de resolução é a necessidade de processamento computacional de grande porte, o que na maioria das vezes leva tais problemas a serem considerados intratáveis, um problema é dito intratável quando não existe um algoritmo executável em tempo polinomial que o resolva. Nesse contexto, inúmeras pesquisas têm se dedicado ao desenvolvimento de técnicas que visam facilitação da modelagem, e principalmente a resolução destes  problemas 
